\documentclass{article}

\usepackage{titlesec}
\newcommand{\sectionbreak}{\clearpage}

\usepackage{fancyhdr}
\pagestyle{fancy}
\lhead{Emanuel Casiano-Diaz}
\rhead{CSYS300: PoCS - Homework 07 - 10/26/2018}
\renewcommand{\headrulewidth}{0.4pt}
\renewcommand{\footrulewidth}{0.4pt}

\usepackage{amsmath}
\usepackage{amssymb}
\usepackage{bm}
\usepackage{pdfpages}

\usepackage{enumerate}% http://ctan.org/pkg/enumerate

\usepackage{hyperref}
\hypersetup{
    colorlinks=true,
    linkcolor=blue,
    filecolor=magenta,      
    urlcolor=cyan,
}

\usepackage{booktabs,float,siunitx}
%\usepackage[demo]{graphicx} % omit 'demo' option in real document

\begin{document}

\section{Exercise 1}

a) $p(x) = cx^{-(q+1)}$ \\

\begin{align}
p_{\geq}(x) &= c \int_{x}^{\infty} t^{-(q+1)}dt \\
&= c \frac{t^-q}{-q} \vert_{t=x}^{t=\infty} \\
p_{\geq}(x) &= \frac{c}{q} x^{-q}
\end{align}

For simplicity, let $\frac{c}{q}\equiv1$. Recall also that:

\begin{equation}
P_{\geq}(A) = \int_{p^{-1}(A^{-\gamma})}^{\infty} p(x) dx = p_{\geq}(p^{-1}(A^{-\gamma}))
 \end{equation}
 
 The inverse function of $p(x) = cx^{-(q+1)}$ is $p^{-1}(x) = x^{-(\frac{1}{q+1})}$. Thus,
 
 \begin{align}
 P_{\geq}(A) &= (p^{-1})^{-q} \\
 &= (x^{-(\frac{1}{q+1})})^{-q} \vert_{x=A^{-\gamma}} \\
 P_{\geq}(A) &= A^{-\gamma(\frac{q}{q+1})} 
 \end{align}
 
 b) $p(x) = ce^{-x}$ \\
 
 \begin{align}
p_{\geq}(x) &= c \int_{x}^{\infty} e^{-t}dt \\
&= -c e^{-t} \vert_{t=x}^{t=\infty} \\
p_{\geq}(x) &= ce^{-x}
\end{align}

For simplicity, let $c \equiv 1$. The inverse function of $p(x) = e^{-x}$ is $p^{-1}(x) = -\log{x}$. Thus, 

\begin{align}
P_{\geq}(A) &= e^{-p^{-1}(A^{-\gamma})} \\
&= e^{\log{A^\gamma}} \\
P_{\geq}(A) &= A^{-\gamma}
\end{align}

c) $p(x) = e^{-x^2}$ \\

The Complementary Cumulative Distribution Function (CCDF) is:

\begin{equation}
p_{\geq}(x) = x^{-1}e^{-x^2} 
\end{equation}

(For full derivation, click here: INSERT LINK). \\

The inverse function of $p(x) = e^{-x^2}$ is $p^{-1}(x) = \sqrt{-\log{x}}$. Thus:

\begin{align}
P_{\geq}(A) &= (p^{-1})^{-1}e^{-(p^{-1})^2} \\
&= (-\log{x})^{-1/2}e^{\log{x}} \vert_{x = A^{-\gamma}} \\
P_{\geq}(A) &= A^{-\gamma}[\log{A}]^{-1/2}
\end{align}

\section{Exercise 2}

The cost function for our discrete H.O.T is:

\begin{equation}
C_{fire} = K \sum_{i=1}^{N_{sites}} p_i a_i
\end{equation}

And the constraint function is:

\begin{equation}
C_{firewalls} = D \sum_{i=1}^{N_{sites}} a_i^{\frac{(d-1)}{d}} a_i^{-1}
\end{equation}

where $K$ and $D$ are proportionality constants. \\

Via the Lagrange Multiplier method, we get the equation:

\begin{equation}
\frac{\partial C_{fire}}{\partial a_i} = \lambda \frac{\partial C_{firewalls}}{\partial a_i}
\end{equation}

Substituting the expressions for $C_{fire}$ and $C_{firewalls}$ into the equation above:

\begin{equation}
K \sum_{i=1}^{N_{sites}} p_i = \lambda D \sum_{i=1}^{N_sites} \frac{d}{da_i}[a_i^{(\frac{d-1}{d})}a^{-i}]
\end{equation}

Applying the product rule of derivatives to the right hand side and dropping the summations:

\begin{equation}
K p_i = \lambda D (\frac{-1}{d} a_i^{\frac{-1}{d}-1})
\end{equation} \\

which implies that:

\begin{equation}
p_i = \lambda \frac{D}{K}  (\frac{-1}{d} a_i^{\frac{-1}{d}-1})
\end{equation}

\begin{equation}
\therefore p_i \propto a_i^{-(1+\frac{1}{d})}
\end{equation}

\section{Exercise 3}

a) The probability of a starting a forest fire at site $(i,j)$ is:

\begin{equation}
P(i,j) = c e^{-i/\ell}e^{-j/\ell}
\end{equation}

where $c$ is a normalization constant. Let the characteristic scale be $\ell = \frac{L}{10}$, where $L$ denotes the linear size of the square lattice. Summing $P(i,j)$ over all possible lattice sites, the normalization constant can be determined:

\begin{align}
1 &= \sum_{i=1,j=1}^{L} c e^{-i/\ell}e^{-j/\ell} \\
&= c \sum_{i=1}^{L} e^{-i/\ell} \sum_{j=1}^{L} e^{-j/\ell} \\
&= c (-\frac{1-e^{-\frac{L}{\ell}}}{1-e^{\frac{1}{\ell}}}) (-\frac{1-e^{-\frac{L}{\ell}}}{1-e^{\frac{1}{\ell}}})
1 &= c (-\frac{1-e^{-\frac{L}{\ell}}}{1-e^{\frac{1}{\ell}}})^2
\end{align}

Solving for $c$ and then substituting $\ell = \frac{L}{10}$, the normalization constant becomes:

\begin{equation}
c = (\frac{1-e^{\frac{10}{L}}}{1-e^{-10}})^2
\end{equation}

b) Code used can be found here: 

\end{document}