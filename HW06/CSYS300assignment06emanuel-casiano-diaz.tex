
\documentclass{article}

\usepackage{titlesec}
\newcommand{\sectionbreak}{\clearpage}

\usepackage{fancyhdr}
\pagestyle{fancy}
\lhead{Emanuel Casiano-Diaz}
\rhead{CSYS300: PoCS - Homework 05 - 10/05/2018}
\renewcommand{\headrulewidth}{0.4pt}
\renewcommand{\footrulewidth}{0.4pt}

\usepackage{amsmath}
\usepackage{amssymb}
\usepackage{bm}
\usepackage{pdfpages}

\usepackage{enumerate}% http://ctan.org/pkg/enumerate

\usepackage{hyperref}
\hypersetup{
    colorlinks=true,
    linkcolor=blue,
    filecolor=magenta,      
    urlcolor=cyan,
}

\usepackage{booktabs,float,siunitx}
%\usepackage[demo]{graphicx} % omit 'demo' option in real document

\begin{document}

\section{Exercise 1}
\subsection{a)}
A forest of size $\ell$ occurs when there the following patter occurs:
\begin{align}
\text{EMPTY}-\underbrace{\text{TREE-TREE . . . -TREE}}_{\ell-times}-\text{EMPTY}
\end{align}

The probability of a tree on site and of an empty site are:
\begin{align}
p \text{ and } (1-p),
\end{align}
respectively.

Thus, the probability of landing in an $\ell-$sized cluster of trees (or forest) is:

\begin{align}
n_{\ell}(p) = (1-p) * \underbrace{p * p * p * \dots p}_{\ell-times} * (1-p)
\end{align}

Or

\begin{align}
n_{\ell}(p) = (1-p)^2 p^{\ell}
\end{align}

\subsection{b)}

Recall that: $0 \leq p \leq 1$.

Percolation in a $1D$ lattice occurs when all sites have trees. The probability of percolation is thus $p^L$, where $L$ is the lattice size. In the limit of $L\to\infty$, then it is seen that:

\begin{align}
\lim_{L\to\infty} p^L = \begin{cases} 0 , 0\leq p < 1 \\1 , p=1 \end{cases}
\end{align}

Thus, for an infinitely large lattice in $1D$, percolation only occurs if $p=1$.

\begin{align}
\therefore p_c = 1
\end{align}

\section{Exercise 2}

Via Real Space Renormalization (RSR), each triangular sublattice is replaced by a single site, or supersite. If the majority of the sites in the sublattice is open (closed), then the supersite is open (closed). The probability that a site is closed is given by $p$ and $(1-p)$, if open. Thus, we need the count how many configurations of triangular sublattices have a majority of closed sites (i.e, 2 or 3 closed) and determine the probability that we will get one of these majority closed sites. Let the majority closed probability be defined as $p'$. Then, applying the discussed RSR rules to the infinitely large triangular lattice, the majority closed probability is:

\begin{align}
p' = p^3 + 3p^2(1-p)
\end{align}

Now, setting $p'=p$, the critical probability can be determined:

\begin{align}
p' &= p \\
\implies p' - p &= 0 \\
2p^3 - 3p^2 + p &= 0 \\
p(p-\frac{1}{2})(p-1) &= 0
\end{align}

Thus: $p=0,1,\frac{1}{2}$. The first two options are just the trivial solutions discussed in the hint video. Thus, the critical probability for the infinitely large triangular lattice is:

\begin{align}
p_c = \frac{1}{2}
\end{align}

\section{Exercise 3}

The code developed to simulate the 2D Percolation Problem can be found here: \url{}


\section{Exercise 4}
\section{Exercise 5}
\end{document}